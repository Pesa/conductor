\documentclass{beamer}
\usepackage{tikz}
\usetikzlibrary{backgrounds}
\usetikzlibrary{shapes}
\usetikzlibrary{positioning}

\usetheme{Frankfurt}
\setbeamertemplate{navigation symbols}{}

\title{A dynamic audio routing system\\based on the RSSI of a Bluetooth device}
\author{Davide Pesavento, Martina Astegno}
\date{\today}

\begin{document}

\begin{frame}[plain]
\titlepage
	\begin{center}
		Universit\`{a} degli Studi di Padova \\
		Corso di Laurea Magistrale in Informatica
	\end{center}
\end{frame} 
 
\begin{frame}
	\frametitle{Outline}
	\tableofcontents 
\end{frame}

%_____________________%

\section{Introduction}

	\begin{frame}
		\frametitle{Project's aim}
		Our application, named \textbf{Conductor}, seeks to achieve two different goals: 
		\pause
		\begin{itemize}
		\item first to locate a person among a set of possible rooms using Bluetooth technology;
		\pause
		\item secondly to redirect audio stream throught different speakers, placed in each room, according to person's movements. 
		\end{itemize}    
	\end{frame}
	
	\begin{frame}
		\frametitle{Tecnologies adopted}
		We adopted two principal tecnologies:
		\pause
		\begin{itemize}
		\item \textbf{Bluetooth} technology for short range wireless communications. We adopted BlueZ as Bluetooth stack implementation.\\ \textbf{HCI} status parameter $\rightarrow$ RSSI value to detect the nearest adapter to the person. 
		\pause
		\item \textbf{Pulseaudio}, a sound server that acts as proxy for sound applications. It redirects audio streams according to person's movements or allows simoultaneos playbacks by more speakers. 
		
		\end{itemize}
	\end{frame}
	
	\begin{frame}
		\frametitle{Hardware requirements}
		There are some basic hardware requirements needed in this context:
		\pause
		\begin{itemize}
		\item \textbf{Smartphone}, to identify the room in which the person is.
		\item \textbf{Central server}, to host the main control application.
		\end{itemize}
		\pause
		And for each rooms:
		\begin{itemize}
		\item a \textbf{speaker} that lets out sound when selected for playback;
		\item a \textbf{Bluetooth adapter} that detects the RSSI value and sends it to the server.
		\end{itemize}
		
		 
		
	\end{frame}
	
%_____________________%

\section{Design and implementation}

	\begin{frame}
	\frametitle{Main project modules}
		In our project we can identify two main modules:
		\begin{itemize}
		\item \textbf{Bluetooth module}
		\item \textbf{Pulseaudio module}
		\end{itemize}
	\end{frame}

	\begin{frame}
		\frametitle{Bluetooth module}
		\textbf{Probe:} a simple application implemented for each Bluetooth adapter. It can interface with BlueZ throught DBus. It interacts with Conductor application throught the \texttt{ProbeInterface} class. This interface can: 
		\begin{itemize}
		\item start/stop device discovery;
		\item update RSSI values when a change occurs.
		\end{itemize}
		\vspace{0.5cm}
		The \texttt{ProbeManager} class controls the Bluetooth operations, in particular:
		\begin{itemize}
		\item monitors device localization;
		\item stores information about every \texttt{ProbeInterface}.
		\end{itemize}
		%The main HCI commands used in our implementation are:
		%\begin{itemize}
		%\item Link Control Commands, to search for other devices in range and to establish connection to other devices;
		%\item Status Parameters Commands, to extract detailed information from the remote device. This includes the link quality and
          %the RSSI value.
		%\end{itemize}
		%The choice of the speaker that has to playback sound is based on the RSSI value provided by Status Parameters commands.
	
	\end{frame}
	
	\begin{frame}
		\frametitle{Pulseaudio module}
		The main class implemented to monitor Pulseaudio operations is \texttt{PAController} class. Each Pulseaudio operation is modeled by one of \texttt{PAOperation} subclasses. \\In particular we focus on these subclasses:

		\begin{itemize}
		\item \texttt{MoveOperation} - it redirects stream audio to one or more different sinks.
		\item \texttt{LoadModuleOperation} - it performs the loading of Pulseaudio modules:
			\begin{itemize}
			\item \texttt{module-tunnel-sink} (tunnel connection with central server);
			\item \texttt{module-combine} (simultaneous playback).
			\end{itemize}
		\end{itemize}
	
	\end{frame}
	
%_____________________%


	\begin{frame}
		\frametitle{General architecture}
		%\mbox{
		\begin{center}

		%\begin{minipage}[c]{5cm}
		\includegraphics[scale = 0.25]{../Arch}
		\end{center}
		%\setlength{\unitlength}{1cm}
		%\onslide<1>
		%\begin{picture}(0,0)(0,0)
		%\color{red}
		%\linethickness{1.5pt}
		%\put(0,3){\line(1,0){3}}
		%\put(0,5){\line(1,0){3}}
		%\put(0,3){\line(0,1){2}}
		%\put(3,3){\line(0,1){2}}
		%\end{picture}
		%\onslide<2->
		%\begin{picture}(3,2)(0,0)
		%\color{red}	
		%\put(2,5){\line(1,0){3}}
		%\put(2,3){\line(1,0){3}}
		%\put(2,3){\line(0,1){2}}
		%\put(5,3){\line(0,1){2}}
		%\end{picture}
		%\begin{picture}(0,0)
		%\linethin{1.5pt}
		%\linethickness{1.5pt}
		%\color{red}
		%put(3,3){\oval(2,3)}
		%\end{picture}
	
		%\setlength{\unitlength}{1cm}
		\vspace*{-8cm}\hspace{0cm}%
		\onslide<1>
		\begin{tikzpicture}[show background rectangle]
		\color{red}
       	\draw[fill=none, line width=0.3mm] (5,10) ellipse (10mm and 3mm);
     	\onslide<2>
       	\draw[fill=none, line width=0.3mm] (0,0) ellipse (10mm and 3mm);
     	\onslide<3->
       	\draw[fill=none, line width=0.3mm] (0,0) ellipse (10mm and 3mm);
     	\end{tikzpicture}
     	


		
		
		
		
		
		
		
		
		
		%\end{minipage}
		%\quad	
		%\begin{minipage}[c]{5cm}
		%\begin{enumerate}
		%\item Periodically \textbf{Probes} obtain RSSI value (INQUIRY)
		%\pause
		%\item If difference between new and old values higher than threshold t $\rightarrow$ update \texttt{RssiModel}
		%\pause
		%\item Algorithm runs $\rightarrow$ output
		%\pause
		%item If output differs from current room $\rightarrow$ send request to \texttt{PAController}
		%\pause
		%\item \texttt{PAController} updates sinks according to algorithm output
		%\end{enumerate}
		%\end{minipage}	
		%}
		
	\end{frame}
	
	\begin{frame}
		\frametitle{Testbed assessment}
	\end{frame}
	
%______________________%

\section{Future work and possible extensions}

	
	\begin{frame}
		\frametitle{Future extensions}
		\begin{itemize}
		\item \textbf{Real world testing:} the prototype has been tested in a simulated environment disregarding many real conditions. It is advisable to perform a more through testing session in a real-world scenario. 
		\pause
		\item \textbf{Multiple devices managing:} it could be interesting to consider a context with more than one device to locate. This could entail problems with devices interaction that have to be considered.
		\pause
		\item \textbf{Control's user addition:} a mobile application that allows to the user to control the audiostream (adjust volume, stop/pause, next/previous, \ldots) could increase user interaction.
		\end{itemize}
	\end{frame}
	
%______________________%

\section{Conclusions}
	\begin{frame}
		\frametitle{Final considerations***TODO***}
	\end{frame}


\end{document}
