\documentclass[conference]{./IEEEtran}

\usepackage{cite}
\usepackage[pdftex]{graphicx}


% correct bad hyphenation here
\hyphenation{op-tical net-works semi-conduc-tor}


\begin{document}

\title{Conductor.....}

\author{\IEEEauthorblockN{Davide Pesavento}
\IEEEauthorblockA{Universit\`{a} degli Studi di Padova\\
Corso di Laurea Magistrale in Informatica\\
Email: email@studenti.math.unipd.it}
\and
\IEEEauthorblockN{Martina Astegno}
\IEEEauthorblockA{Universit\`{a} degli Studi di Padova\\
Corso di Laurea Magistrale in Informatica\\
Email: email@studenti.math.unipd.it}}

% make the title area
\maketitle


\begin{abstract}
%\boldmath
We present in this paper an application that shows how the bluetooth technology can be used to locate a person who has a bluetooth device (e.g. smartphone) in a set of possible rooms.  The final aim is to playback an audio stream stored in a central server and move it from a speaker to an another, situated in every room, following person's movements.
\end{abstract}
% IEEEtran.cls defaults to using nonbold math in the Abstract.
% This preserves the distinction between vectors and scalars. However,
% if the conference you are submitting to favors bold math in the abstract,
% then you can use LaTeX's standard command \boldmath at the very start
% of the abstract to achieve this. Many IEEE journals/conferences frown on
% math in the abstract anyway.

\begin{IEEEkeywords}%normally used for peerreview paper
Bluetooth, Pulseaudio, conductor.
\end{IEEEkeywords}




% For peer review papers, you can put extra information on the cover
% page as needed:
% \ifCLASSOPTIONpeerreview
% \begin{center} \bfseries EDICS Category: 3-BBND \end{center}
% \fi
%
% For peerreview papers, this IEEEtran command inserts a page break and
% creates the second title. It will be ignored for other modes.
\IEEEpeerreviewmaketitle


%\hfill October 16, 2010

\section{Introduction}
In 1994 L.M. Ericsson company began to be interested in wireless connection between cellular phones and other devices. With other four companies it formed a group, SIG, to develop a new standard wireless used to link computation and communication devices throught a wireless radio system with low costs, restricted power and range. The project was called Bluetooth and after five years it was published the first version 1.0 specific.
In figure \ref{stack} we can see the architecture of bluetooth protocol.
\begin{figure}[h]
\centering
\includegraphics[scale = 0.22]{stackBT.png}
\caption{Bluetooth Protocol Stack}
\label{stack}
\end{figure}

The protocols can be divided in four major classes:
\begin{itemize}
\item Core protocols
\item Cable replacement protocols
\item Telephony control protocols
\item Adopted protocols
\end{itemize}

-------TOFIX: a questo punto pesanvo di concertrare l'attenzione sul service discovery protocol usato per individuare le altre devices.e spiegare u pò come funzia


The bluetooth protocol acts in 2.4Ghz frequency using 79 channels (between 2.402 and 2.480 Ghz) and the maximum transmission velocity is 2.1 Mbit/s (version 2.0). The devices can be divided in three classes according to power level and range.\\
dopo un intro alla tecnologia del bluetooth introduco un pò coem viene gestito il flusso audio con pulseaudio server e così finisce l'intro-----------
\section{General architecture}
\subsection{Requirements}
\begin{itemize}
\item antenne+speaker per ogni room
\item smartphone (1)
\item central server
\end{itemize}
%immagine con la architettura ad alto livello che mostra come interagiscono le componenti

\subsection{Bluetooth module}
\subsection{Pulseaudio module}
\subsection{Algorithm}



\section{Runtime behavior}
\subsection{Configuration}
\subsection{Monitoring}


\section{Experiments Evaluation}

\section{Problem handling}

\section{Future work and conclusion}
The conclusion goes here.




% conference papers do not normally have an appendix


% use section* for acknowledgement
%\section*{Acknowledgment}


%The authors would like to thank...





% trigger a \newpage just before the given reference
% number - used to balance the columns on the last page
% adjust value as needed - may need to be readjusted if
% the document is modified later
%\IEEEtriggeratref{8}
% The "triggered" command can be changed if desired:
%\IEEEtriggercmd{\enlargethispage{-5in}}

% references section

% can use a bibliography generated by BibTeX as a .bbl file
% BibTeX documentation can be easily obtained at:
% http://www.ctan.org/tex-archive/biblio/bibtex/contrib/doc/
% The IEEEtran BibTeX style support page is at:
% http://www.michaelshell.org/tex/ieeetran/bibtex/
%\bibliographystyle{IEEEtran}
% argument is your BibTeX string definitions and bibliography database(s)
%\bibliography{IEEEabrv,../bib/paper}
%
% <OR> manually copy in the resultant .bbl file
% set second argument of \begin to the number of references
% (used to reserve space for the reference number labels box)
\begin{thebibliography}{1}
\bibitem{Pulseaudio} PROVA
Pulseaudio
\bibitem{Qt}
\bibitem{altro}

%\bibitem{IEEEhowto:kopka}
%H.~Kopka and P.~W. Daly, \emph{A Guide to \LaTeX}, 3rd~ed.\hskip 1em plus
 % 0.5em minus 0.4em\relax Harlow, England: Addison-Wesley, 1999.

\end{thebibliography}




% that's all folks
\end{document}


